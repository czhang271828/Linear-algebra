\documentclass[11pt]{ctexart}
\usepackage[margin=2cm,a4paper]{geometry}
\usepackage{amsthm, amsfonts, amsmath, amssymb, mathrsfs, newclude, tikz-cd, tikz, ctex, mathtools, stmaryrd, datetime}

%\setmainfont{Caladea}

%% 也可以选用其它字库:
% \setCJKmainfont[%
%   ItalicFont=AR PL KaitiM GB,
%   BoldFont=Noto Sans CJK SC,
% ]{Noto Serif CJK SC}
% \setCJKsansfont{Noto Sans CJK SC}
% \renewcommand{\kaishu}{\CJKfontspec{AR PL KaitiM GB}}



\usepackage[colorlinks = true,
linkcolor = blue,
urlcolor  = blue,
citecolor = blue,
anchorcolor = blue]{hyperref}

% Include the x-color package for color support
\usepackage{xcolor}

% Define a new environment for red comments
\usepackage{verbatim} % Required for the comment environment
\usepackage{environ}

\usepackage{mdframed} % Include mdframed for creating framed environments

\definecolor{pinked}{RGB}{255,231,229} % Define a base color 
% Define a new environment with a background color
\newmdenv[
  backgroundcolor=pinked, % Set the desired background color
  linecolor=white, % Optional: Set the border line color
  linewidth=1pt, % Optional: Set the border line width
  roundcorner=5pt, % Optional: Set rounded corners
  nobreak=true % Optional: Prevent page breaks within the environment
]{pinked}

\theoremstyle{definition}
\newtheorem{qqq}{问题}[section]

\newcommand{\ExternalLink}{%
    \tikz[x=1.2ex, y=1.2ex, baseline=-0.05ex]{% 
        \begin{scope}[x=1ex, y=1ex]
            \clip (-0.1,-0.1) 
                --++ (-0, 1.2) 
                --++ (0.6, 0) 
                --++ (0, -0.6) 
                --++ (0.6, 0) 
                --++ (0, -1);
            \path[draw, 
                line width = 0.5, 
                rounded corners=0.5] 
                (0,0) rectangle (1,1);
        \end{scope}
        \path[draw, line width = 0.5] (0.5, 0.5) 
            -- (1, 1);
        \path[draw, line width = 0.5] (0.6, 1) 
            -- (1, 1) -- (1, 0.6);
        }
    }

\NewEnviron{aaa}{~\\
    \noindent {\textcolor{teal}{\textbf{解答}} \BODY }
}

\NewEnviron{llll}{
    \noindent {~\\$\ExternalLink$ 外部链接 $\,\,\,$ \color{blue}\url{\BODY} }
}

\renewcommand{\proofname}{证明}
\renewcommand\qedsymbol{${\boxed{\substack{\textit{完证}\\\textit{毕明}}}}$}
\let\oldproof\proof
\renewcommand{\proof}{\color{blue}\oldproof}

% Change equation numbering to include the section number
\usepackage{cleveref}
\renewcommand{\theequation}{\thesection.\thesubsection.\arabic{equation}}
\numberwithin{equation}{section}

\usepackage{listings}
% Define listings style
\lstset{
  frame=tb,
  language=TeX,
  aboveskip=3mm,
  belowskip=3mm,
  showstringspaces=false,
  columns=flexible,
  basicstyle={\small\ttfamily},
  numbers=none,
  breaklines=true,
  breakatwhitespace=true,
  tabsize=3
}

\theoremstyle{definition}
\newtheorem*{definition}{定义}
\newtheorem*{proposition}{命题}
\newtheorem*{theorem}{定理}
\newtheorem*{notation}{记号}
\newtheorem*{example}{例子}
\newtheorem{exercise}{\textcolor{blue}{习题}}
\theoremstyle{remark}
\newtheorem*{remark}{备注}
\newtheorem*{lemma}{引理}
\newtheorem*{corollary}{推论}

\title{习题: 对称矩阵 (\LaTeX 重排)}
\author{(2024-2025-1)-MATH1405H-02}

\setcounter{page}{0}

\setlength\parindent{0pt}

\begin{document}

\maketitle

\vspace{5cm}

\section*{前言}

\begin{pinked}
\begin{theorem}[相抵标准型]
    对任意 $A\in \mathbb F^{m\times n}$, 存在 $P\in \mathrm{GL}_m(\mathbb F)$ 与 $Q\in \mathrm{GL}_n(\mathbb F)$ 使得 
\begin{equation}
    PAQ=\begin{pmatrix}I_r&O\\O&O\end{pmatrix}\qquad( r=\mathrm{rank}(A)).
\end{equation}
\end{theorem}
\end{pinked}

\begin{remark}
    注: $r$ 唯一确定. $P$ 和 $Q$ 不必唯一. 
\end{remark}

\begin{remark}
    域是任意的. 
\end{remark}

\begin{remark}
$\mathrm{GL}_n(\mathbb F)$ 代表 $\mathbb F$ 上 $n$-阶可逆矩阵全体 (一般作为乘法群). 
\end{remark}

\begin{remark}
    矩阵的行列不必等长. 方阵的行列等长. 
\end{remark}

\begin{remark}
    记 $r(A)$ 为矩阵 $A$ 的秩. 
\end{remark}

\begin{pinked}
    请不要盲目使用 Jordan 标准型. 在完整地引入域扩张, 初等因子等理论之前, Jordan 标准型仅对 $\mathbb C$ 上的矩阵有效. 此处的 $\mathbb F$ 是任意域. 
\end{pinked}

\newpage

\section{第一部分习题}

\begin{exercise}[同时相抵化] 对相同规格的矩阵 $A$ 与 $B$. 若
    \begin{equation}
        \mathrm{rank}(A+B)=\mathrm{rank}(A)+\mathrm{rank}(B),
    \end{equation}
    则存在 $P\in \mathrm{GL}_m(\mathbb F)$ 与 $Q\in \mathrm{GL}_n(\mathbb F)$ 使得 
    \begin{equation}
        PAQ=\begin{pmatrix}I_{\mathrm{rank}(A)}&O&O\\O&O&O\\O&O&O\end{pmatrix},\quad PBQ=\begin{pmatrix}O&O&O\\O&O&O\\O&O&I_{\mathrm{rank}(B)}\end{pmatrix}.
    \end{equation}
\end{exercise}

\begin{exercise}[分块上三角化] 
    记矩阵 (各分块不必是方阵)
\begin{equation}
    M=\begin{pmatrix}A&C\\O&B\end{pmatrix}.
\end{equation} 
证明 $r(M)=r(A)+r(B)$ 的充要条件如下: 
\begin{itemize}
    \item 存在矩阵 $X$ 与 $Y$ 使得 $AX+YB=C$.
\end{itemize}
\end{exercise}

\begin{exercise}[何时能砍掉无用的行列空间]
    所有矩阵不必是方阵. 证明: 
    \begin{equation}
        r\left(\begin{pmatrix}
            A&B\\C&D
            \end{pmatrix}\right)=r(A)
    \end{equation}
    的充要条件是存在 $X$ 与 $Y$, 使得以下三个等式同时成立
\begin{equation}
    AX=B,\quad YA=C,\quad YAX=D.
\end{equation}
\end{exercise}

\section{第二部分习题}




\end{document}