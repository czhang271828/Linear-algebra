\documentclass[11pt]{ctexart}

\usepackage[margin=2cm,a4paper]{geometry}
\usepackage{amsthm, amsfonts, amsmath, amssymb, mathrsfs, newclude, tikz-cd, tikz, ctex, mathtools, stmaryrd, datetime}


\usepackage{braket}

%\setmainfont{Caladea}

%% 也可以选用其它字库:
% \setCJKmainfont[%
%   ItalicFont=AR PL KaitiM GB,
%   BoldFont=Noto Sans CJK SC,
% ]{Noto Serif CJK SC}
% \setCJKsansfont{Noto Sans CJK SC}
% \renewcommand{\kaishu}{\CJKfontspec{AR PL KaitiM GB}}



\usepackage[colorlinks = true,
linkcolor = blue,
urlcolor  = blue,
citecolor = blue,
anchorcolor = blue]{hyperref}

% Include the x-color package for color support
\usepackage{xcolor}

% Define a new environment for red comments
\usepackage{verbatim} % Required for the comment environment
\usepackage{environ}

\usepackage{mdframed} % Include mdframed for creating framed environments

\definecolor{pinked}{RGB}{255,231,229} % Define a base color 
% Define a new environment with a background color
\newmdenv[
  backgroundcolor=pinked, % Set the desired background color
  linecolor=white, % Optional: Set the border line color
  linewidth=1pt, % Optional: Set the border line width
  roundcorner=5pt, % Optional: Set rounded corners
  nobreak=true % Optional: Prevent page breaks within the environment
]{pinked}

\theoremstyle{definition}
\newtheorem{qqq}{问题}[section]

\newcommand{\ExternalLink}{%
    \tikz[x=1.2ex, y=1.2ex, baseline=-0.05ex]{% 
        \begin{scope}[x=1ex, y=1ex]
            \clip (-0.1,-0.1) 
                --++ (-0, 1.2) 
                --++ (0.6, 0) 
                --++ (0, -0.6) 
                --++ (0.6, 0) 
                --++ (0, -1);
            \path[draw, 
                line width = 0.5, 
                rounded corners=0.5] 
                (0,0) rectangle (1,1);
        \end{scope}
        \path[draw, line width = 0.5] (0.5, 0.5) 
            -- (1, 1);
        \path[draw, line width = 0.5] (0.6, 1) 
            -- (1, 1) -- (1, 0.6);
        }
    }

\NewEnviron{aaa}{~\\
    \noindent {\textcolor{teal}{\textbf{解答}} \BODY }
}

\NewEnviron{llll}{
    \noindent {~\\$\ExternalLink$ 外部链接 $\,\,\,$ \color{blue}\url{\BODY} }
}

\renewcommand{\proofname}{证明}
\renewcommand\qedsymbol{${\boxed{\substack{\textit{完证}\\\textit{毕明}}}}$}
\let\oldproof\proof
\renewcommand{\proof}{\color{blue}\oldproof}

% Change equation numbering to include the section number
\usepackage{cleveref}
\renewcommand{\theequation}{\thesection.\thesubsection.\arabic{equation}}
\numberwithin{equation}{section}


% 创建适合 python 语言的公式环境. 
\usepackage{pythonhighlight}



\theoremstyle{definition}
\newtheorem*{definition}{定义}
\newtheorem*{proposition}{命题}
\newtheorem*{theorem}{定理}
\newtheorem*{notation}{记号}
\newtheorem*{example}{例子}
\newtheorem{exercise}{\textcolor{blue}{习题}}
\theoremstyle{remark}
\newtheorem*{remark}{备注}
\newtheorem*{lemma}{引理}
\newtheorem*{corollary}{推论}

\title{第六次作业解答}
\author{(2024-2025-1)-MATH1405H-02}

\setcounter{page}{0}

\setlength\parindent{0pt}

\begin{document}

\maketitle

\vspace{5cm}

\newpage

\section{第一题解答}

\begin{exercise}
    给定 $\|u\|\leq 1$ 与 $\|v\|\leq 1$, 证明: $\sqrt{1-\|u\|^2}\cdot \sqrt{1-\|v\|^2}\leq 1-\braket{u,v}$.
    \begin{proof}
        只需证明 $\sqrt{1-\|u\|^2}\cdot \sqrt{1-\|v\|^2}\leq 1-\|u\|\cdot \|v\|$. 平方得 $-\|u\|^2-\|v\|^2\leq -2\|u\|\cdot \|v\|$, 即 $0\leq (\|u\|-\|v\|)^2$. 
    \end{proof}
\end{exercise}  

\begin{exercise}
    证明: $\|u+v\|\cdot \|u-v\|\leq \|u\|^2+\|v\|^2$.
    \begin{proof}
        对两侧平方, 得 $(\|u\|^2+\|v\|^2)^2-4\braket{u,v}^2\leq(\|u\|^2+\|v\|^2)^2$. 
    \end{proof}
\end{exercise}

\begin{exercise}
    证明: $\braket{u,v}=0$, 当且仅当 $\|u\|\leq \|u+c \cdot v\|$ 对一切 $c\in \mathbb R$ 成立.
    \begin{proof}
        ($\to$ 方向) 若 $\braket{u,v}=0$, 往证 $\|u\|\leq \|u+c \cdot v\|$. 等价地, 对待证式两侧平方得
        \begin{equation}
            \|u\|^2\leq \|u\|^2+2c\braket{u,v}+c^2\|v\|^2.
        \end{equation} 
        消元, 并代入正交条件, 上式对一切 $c\in \mathbb R$ 取等. 

        ($\gets$ 方向) 若不等式对一切 $c\in \mathbb R$ 取等, 则右式平方的一次项系数为零, 即 $\braket{u,v}=0$. 
    \end{proof}
\end{exercise}

\begin{exercise}
    证明: 任意给定 $u\neq \mathbf 0$, 则对一切 $\|v\|=1$ 均有 $\|u-(\|u\|^{-1}\cdot u)\|\leq \|u-v\|$. 换言之, 球面上距 $u$ 最近处恰是 $u$ 的单位化向量.
    \begin{proof}
        若空间是 $0$ 维或一维的, 则结论平凡. 不妨假定空间维数不小于 $2$. 以下使用反证法: 
        \begin{itemize}
            \item 若存在 $v\neq \|u\|^{-1}u$ 使得 $\|u-(\|u\|^{-1}\cdot u)\|\leq \|u-v\|$, 则只需要球面与二维空间 $V:=\mathrm{span}(u,v)$ 的交中找到矛盾即可.  
        \end{itemize}
        取 $V$ 的单位正交基 $e_1=\|u\|^{-1}\cdot u$ 与 $e_2$, 则 $v$ 位于圆周 ($V$ 与球面的交). 圆周上的点 $w$ 具有一般形式 $\cos\theta e_1+\sin \theta e_2$, 此时 
        \begin{equation}
            \|u-w\|^2= 1 + \|u\|^{-2}(1-2\cos\theta). 
        \end{equation}
        当 $\|u-w\|$ 取最小值时, 必有 $\cos\theta =1$. 此时 $w=e_1=\|u\|^{-1}u$, 与 $v$ 的选取矛盾. 
    \end{proof}
\end{exercise}

\begin{exercise}
    表述并证明高中所学的极化恒等式.
    \begin{proof}
        图形描述见\href{https://linear.axler.net/LADR4e.pdf#page=209}{教材 LADR 的 195 页}. 这是半侧平行四边形法则. 
    \end{proof}
\end{exercise}

\begin{exercise}
    表述并证明高中所学的平行四边形恒等式.
    \begin{proof}
        记 $u$ 与 $v$ 是内积空间的向量, 则
        \begin{equation}
            2(\|u\|^2+\|v\|^2)=\|u+v\|^2+\|u-v\|^2. 
        \end{equation}
        依定义展开得
        \begin{itemize}
            \item $\|u+v\|^2=\braket{u+v,u+v}=\braket{u,u}+ \braket{u,v}+\braket{v,u}+\braket{v,v}=\|u\|^2+2\braket{u,v}+\|v\|^2$,
            \item $\|u-v\|^2=\braket{u-v,u-v}=\braket{u,u}- \braket{u,v}-\braket{v,u}+\braket{v,v}=\|u\|^2-2\braket{u,v}+\|v\|^2$. 
        \end{itemize}
        相加即可. 
    \end{proof}
\end{exercise}

\section{第二题解答}

\begin{definition}[投影矩阵]
    假定 $n\geq 1$. 称 $P\in \mathbb R^{n\times n}$ 是 $\mathbb R^n$ 上的一个投影矩阵, 当且仅当 $P^2=P=P^T$.
\end{definition}

\begin{exercise}
    使用相抵标准型证明, 若 $P$ 是投影矩阵, 当且仅当存在正交矩阵 $Q$ 使得 $Q^{-1}PQ = \begin{pmatrix}I_r&O\\O&O\end{pmatrix}$. (如果不清楚正交矩阵, 此题可以略过.) 
    \begin{proof}
        使用相抵标准型, 
    \end{proof}
\end{exercise}

2. 证明: 投影矩阵和子空间双射对应, 具体的对应方式可以是列空间 $P\xleftrightarrow {1:1} C(P)$.
3. 证明: 投影矩阵和子空间双射对应, 具体的对应方式可以是零空间 $P\xleftrightarrow {1:1} N(P)$.
4. 任意给定 $v\neq \mathbf 0$, 找到 $P$ 使得 $C(P)=\mathrm{span}(v)$.
5. 任意给定 $v\neq \mathbf 0$, 找到 $P$ 使得 $N(P)=\mathrm{span}(v)$.
6. 给定 $\mathbb R^5$ 中的列向量 $S=\{(4,3,3,1,1),(6,2,2,2,1)\}$, 找到 $P\in \mathbb R^{5\times 5}$ 使得 $C(P)=\mathrm{span}(S)$.
7. 给定 $\mathbb R^5$ 中的列向量 $S=\{(4,3,3,1,1),(6,2,2,2,1)\}$, 找到 $P\in \mathbb R^{5\times 5}$ 使得 $N(P)=\mathrm{span}(S)$.

备注: 计算得 $\|(4,3,3,1,1)\|=6$, 以及 $\|(6,2,2,2,1)\|=7$.



**Challenge** 投影矩阵的和与积都不必是投影矩阵 (实际上, 任何不可逆方阵都是有限个投影矩阵的乘积). 能否优雅地定义投影矩阵间的二元运算 $\sqcap$ 与 $\sqcup$, 使得
$$
C(P_1\sqcup P_2)=C(P_1)+C(P_2),\quad C(P_1\sqcap P_2)=C(P_1)\cap C(P_2).
$$



**Problem 3** 计算示例 (最小平方法).


1. 给定 $A=(a_{i,j})\in \mathbb R^{m\times n}$ 与 $b=(b_j)\in \mathbb R^{n}$. 记 $x=(x_i)\in \mathbb R^m$, 则
    $$
    F=\|Ax-b\|^2=\sum_{i=1}^m\left(\sum_{j=1}^na_{i,j}x_j-b_i\right)^2.
    $$
    任给定指标 $1\leq j_0\leq n$, 假设所有 $x_j$ ($j\neq j_0$) 均是常量, 仅 $x_{j_0}$ 是变量. 通过下式计算二次函数 $F=F(x_{j_0})$ 导数为零的点
    $$
    \frac{\operatorname d F}{\operatorname d x_{j_0}} =\frac{\operatorname d }{\operatorname d x_{j_0}}\left[\sum_{i=1}^m\left(\sum_{j=1}^na_{i,j}x_j-b_i\right)^2\right] =0.
    $$
    * $F$ 何时是二次函数? 我们需要排除一些平凡情形, 请稍作说明.
2. 假设上一问中 $F(x_{j_0})=0$ 的解是 $x_{j_0}=X_{j_0}$. 记解向量 $X=(X_i)\in \mathbb R^m$. 证明 $A^TAX=A^Tb$.
3. (自主思考, 这不是一个问题) 上式合并了有唯一解, 有无穷解, 以及无解这三种情况. 请区分, 讨论这些情况.
4. 使用最小平方法找到一条抛物线 $y=a+bx+cx^2$, 使得该抛物线可以尽可能地拟合以下所有点:
    $$
    \{(-2,4),(-1,2),(0,1),(2,1),(3,1)\}.
    $$
    提示: 可以考虑方程
    $$
    \begin{pmatrix}1&x_1&x_1^2\\1&x_2&x_1^2\\\vdots &\vdots& \vdots \\1&x_n&x_n^2\end{pmatrix}\cdot \begin{pmatrix}a\\ b\\ c\end{pmatrix} =\begin{pmatrix}y_1\\ y_2\\ \vdots \\ y_n\end{pmatrix}.
    $$
    请用严谨的数学语言解释这一所谓的拟合 (应当先定义点到抛物线的距离.).



**Challenge** 先前有一道证明题: $\mathbb R^n$ 的任意有限个真子空间之并不是全空间. 此处有一道类似的问题:
* 任取 $\mathbb R^n$ 中有限个真子空间 $\{V_i\}_{i=1}^m$, 则必存在补集中的向量组 $\{f_i\}_{i=1}^n\subset \left(\bigcup_{i=1}^m V_i\right)^c$, 使得 $\braket{f_i,f_j}=\delta_{i,j}$ (单位正交关系).
* 以上 $\{f_i\}_{i=1}^n$ 有无穷多种取法 (这或许是一句废话.).

\end{document}